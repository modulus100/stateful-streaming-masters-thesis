\section{Recap of the Research}\label{sec:recap-of-the-research}
First at all, it needs to be mentioned that periodic batch processing job
is might not be that scalable and fault-tolerant as modern
stream processing frameworks.
Stream processing can replace and should replace solutions which require
a continuous stream processing with a minimal latency.
Proper configuration and replica size significantly reduces latency comparing
to periodic batch processing solutions.
However, it also helps batch processing to reduce network load.
But having Kafka interface, a requirement in low latency and fault tolerance,
stream processing solutions win.

\section{Concluding Thoughts}\label{sec:concluding-thoughts}
Both Kafka Streams and Apache Flink demonstrate incredible performance
and state recovery in case of replicas restoration.
Apache Flink show better performance for in case of a high load
and cpu extensive operations which makes Flink preferable
option for heavy loaded systems.
With better performance Flink requires deep knowledge in Flink
API and its maintenance and working modes.
It definitely requires a developer to learn additional technologies and documentation.
Flink has an awesome community which is quite helpful with problems,
for example Flink community Slack channel.
Since Apache Flink is a part of Confluent, there are more and more tutorials
being added for Flink which makes it learning more efficient.
Kafka Streams cannot fully replace Flink, and it certainly has its own niche.

Another important difference between Apache Flink and Kafka Streams which
may affect a preference is Flink's ability to work with any kind of data
sources, but Kafka Streams is specifically designed for Kafka.
Apache Flink also has a native Python support, which allows Flink
to be used with existing Python libraries, for almost any use case.

\section{Practical Applications}\label{sec:practical-applications}
The most usable use case for Flink is real time data analytics \cite{flink_use_cases}.
Flink is actively used with real time machine learning algorithms, for
real time predictions having different integration with Python ML libraries
or with Java based FlinkML \cite{flink_ml}.
Both Flink and Kafka Streams could be used for real time Fraud detection
and IoT solutions, processing thousands or even millions of data sources.


\section{Future work}\label{sec:future-work}
Future work as an extension for this thesis is supposed to get benchmarks
for differences state backends, different state sizes, network issues.
Chaos Mesh \cite{chaosMesh} provides different tools for extensive chaos engineering
uses cases.