\section{Related work}\label{sec:interpretation-of-results}
There are lots of informative publications that have been made
regarding stream processing systems.
For example, Theodolite framework \cite{theodolite_framework} was born
due to the need to evaluate performance in big data systems, especially
targeting cloud environment based on Kubernetes.
The framework comes with scalability benchmarks for Apache Flink and Kafka Streams.
The use covers a case where the system must process thousands of sensors in
real-time.
Results show how Apache Flink can be efficient in case of increasing data volumes,
such that Flink requires fewer replicas to handle the same amount of data.
Another study covers benchmarks for windowing aggregations \cite{dataSystemBenchmarks}.
Apache Flink, Apache Storm, and Apache Spark are considered for SQL-based windowing, where Apache Flink shows higher throughput.
The recent release of ShuffleBench \cite{Henning_2024} that is focusing on scalability
benchmarks evaluation of Apache Flink, Hazelcast Jet, Apache Spark, and Kafka Streams.

The following studies \cite{carbone2015lightweight} \cite{siachamis2024checkmate} focus
primarily on checkpoints and state recovery use cases.
The foundation covered in these studies gives a full overview of Flink's performance in
case of a state recovery, especially in case of extensive data streams.

\section{Future work}\label{sec:future-work}
Future work as an extension for this thesis is supposed to get benchmarks
for differences state backends, different state sizes, network issues.
Chaos Mesh \cite{chaosMesh} provides different tools for extensive chaos engineering
uses cases.
