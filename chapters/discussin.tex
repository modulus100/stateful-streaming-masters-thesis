\section{Interpretation of Results}\label{sec:interpretation-of-results}
This research is focusing on 3 metrics, which are input rate, output rate and consumer group lag.
Streaming prototypes which were benchmarked have two main characteristics, which are
input/output interfaces and processing performance.
Such results give a general performance overview of the system.
Further research could be focused on specific metrics.
The reason behind selecting 10000 rules and selectivity of 0.00002 is a
simulation of a real production system.
The main reason for using number of replicas between 10 and 12 for experiments
was to find out a performance difference in case of different Java versions and replicas shortage.
For instance, if one of Java versions would perform significantly better, then
it's possible to reduce a number of replicas for the same workload and save
solution costs.
In ideal solution a consumer group commit lag must never grow but
jumping between 0 and n, where n.
The higher n is, the bigger delay is as well.
For a real production systems, commit interval and number of replicase
obviously will have to be tuned depending on the requirements.
If the system is able to process the load, then consumer group lag never grow.
But in order to avoid 30 seconds delay, even if the system performs well,
it's important to adjust Kafka commit period, for example make it 10 or 20 seconds.

Fault tolerance benchmark was done with Java 21, because such metric is
focused on general system behaviour of a state recovery time rather than on Java version,
however there is still could be some performance difference.
The main metric of a state recovery time could be defined as a period of a time,
from replica kill to a start of stabilized consumer group commit, when it doesn't
change anymore.

Another motivational aspect is a release of Java 21.
It's curious if Java 21 could produce a significant performance difference.

The biggest different for Apache Flink and Kafka Streams implementation is output.
Apache Flink sends each processed event to the output topic right after process single record,
whereas Kafka Streams implementation aggregates many records first and sends aggregated
records to the output topic.
This is a main reason why Kafka Streams has less MPS in output, and they get send
in interval whereas Apache Flink streams them constantly, without periods.
Both solutions use at least once semantics, which is a default.

\section{Comparison with Existing Literature}\label{sec:comparison-with-existing-literature}
Experiments with the matched service are also been researched by Dynatrace research
group in collaboration with JKU University. \cite{kleppmann2017}, \cite{kafka2020}, \cite{confluent2023}
The research could be found here \cite{ICPE2024}.

\section{Implications of Findings}\label{sec:implications-of-findings}
Discuss the broader implications of your results, particularly in the context of Apache Flink and Kafka Streams.

\section{Limitations of the Study}\label{sec:limitations-of-the-study}
Acknowledge any limitations in your research approach or data

\section{Suggestions for Future Research}\label{sec:suggestions-for-future-research}
Offer recommendations for further study in this area.